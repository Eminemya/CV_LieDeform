\section{Related Work}
\label{sec:review}

We first briefly review previous work on deformations, which roughly
fall into two categories.
The first category of methods focuses on estimating global affine
transformation. Frey and Jojic~\cite{TMG1,TMG2,TMG3} proposes a
mixture model, where the space of affine transforms is discretized,
and an indicator is used to choose a specific transform in generating
each image.
Miller~\etal~\cite{Congeal} proposed a nonparametric probabilistic
model, which estimates the global affine transforms by gradually
aligning the images, using a gradient descent method referred to as ``congealing''.

The second category takes into account non-rigid deformations that can
lead to changes in shapes. Cootes~\etal~\cite{AAM,JP} proposed the
active appearance model for object alignment, where the deformation is
represented via the displacement of pre-specified control points.  In
addition, approaches by directly matching local descriptors are also
widely used.  Belongie et al.~\cite{SC} developed a direct matching
method using local shape context based on statistics of edges.  Keyser
et al.~\cite{IDM} developed an Image Distortion Model
(IDM)~\cite{IDM}, which pursues a dense match of local patches between
two images as a representation of the deformation.  Though simple,
this method leads to substantial improvement on character recognition,
providing significant evidence as to the important role of local
deformations in object recognition.  The pioneering work by
Tenenbaum~etal~\cite{ISO} and Roweis and Saul~\cite{LLE} initiated a
large amount of work that directly models the image manifold via
embeddings it into local low-dimensional spaces.

While deformation information is made use of in building object
metrics in the work mentioned above, these methods do not establish
an explicit model of deformations. Recently, new models have been
proposed to address this issue.
Simard et al.~\cite{TD} considered the manifold of deformations,
approximating it via local tangent spaces. In this work, the
basis of these tangent spaces are hand-crafted, with some apparent
deformation patterns taken into account (\eg~rotation and changes of
thickness). However, some subtle variations of shapes are difficult to
capture via manually devised patterns.
Another drawback of this method lies in the introduction of tangent
spaces for all training samples, incurring unnecessary
computational costs in both training and testing phases when the
samples are dense.
%
Hastie and Simard~\cite{HTrevor,ATV} improve upon this method by
grouping nearby samples into clusters and deriving the tangent basis
via learning. However, learning is performed independently for each
tangent space, utilizing only the samples within a local neighborhood.
This makes it potentially difficult to obtain reliable estimations. 
%%% Local Variables:
%%% TeX-master: "main_paper"
%%% End:







